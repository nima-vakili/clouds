\documentclass[11pt]{article}

\usepackage{geometry}
\geometry{a4paper}

\usepackage{graphicx} 
\usepackage{natbib} 
\usepackage{amsmath,amsthm,amssymb,latexsym,amsfonts}
\usepackage{bm}
\usepackage{sectsty} 
\usepackage{color}

\sectionfont{\sffamily\bfseries\upshape\large}
\subsectionfont{\sffamily\bfseries\upshape\normalsize}
\subsubsectionfont{\sffamily\em\upshape\normalsize}

\newcommand{\KL}[2]{D(#1\|#2)}
\newcommand{\hide}[1]{{}}
\newcommand{\innerprod}[2]{{#1}^\mathrm{T}{#2}}
\newcommand{\outerprod}[2]{{#1}{#2}^\mathrm{T}}
\newcommand{\covariance}[1]{\mathrm{cov}(#1)}
\newcommand{\features}{\bm{f}}
\newcommand{\state}{\bm{x}}
\newcommand{\population}{\bm{X}}
\newcommand{\trace}{\mathrm{tr}}
\newcommand{\params}{\bm{\lambda}}
\newcommand{\protocol}{\bm{\Lambda}}
\newcommand{\gradient}{\bm{\partial}}

\renewcommand{\innerprod}[2]{\langle{#1}|{#2}\rangle}
\renewcommand{\outerprod}[2]{|{#1}\rangle\langle{#2}|}

\renewcommand{\innerprod}[2]{{#1}^{\bm{\prime}}\!{#2}}
\renewcommand{\outerprod}[2]{{#1}{#2}^{\bm{\prime}}}

\begin{document}

\noindent\textsf{\textbf{\Large Gibbs Sampling algorithm for GMM with unknown Rotation}}

\vspace*{1cm}\noindent{\sffamily\bfseries Michael Habeck $^{1,2}$, Nima~Vakili$^{2}$}\\

\noindent{\sffamily\footnotesize%
%
\textbf{1-} Statistical Inverse Problems in Biophysics, Max Planck Institute for Biophysical Chemistry, Am Fassberg 11, 37077 G\"ottingen, Germany\\
\textbf{2-} Felix Bernstein Institute for Mathematical Statistics in the Biosciences, Georg August University G\"ottingen, Germany}
%
\\ \\

\section*{Computing posterior by using conditional and prior states}
The full posterior for Gibbs sampler is:
\begin{equation*}
p(\theta,Z | D)\propto p(D,Z|\theta)p(\theta)
\end{equation*}
where the full likelihood would be:
\begin{equation*}
p(D,Z|\theta)=p(x,z|\mu,\tau,\omega,R)
\end{equation*}
and 
\begin{equation*}
p(\theta)\propto p(\mu|\tau)p(\tau)p(\omega)p(R)
\end{equation*}\\\\
First we have to compute the whole priors.\\
Prior for the means is: \\
\begin{eqnarray*}
p(\mu|\tau) \propto \prod_{k=1}^K p(\mu_k|\tau) &=& \prod_{k=1}^K N(\mu_k|\mu_0,\tau\tau_0I) \\
&=& \prod_{k=1}^K\frac{\tau\tau_0}{2\pi}^{\frac{D+1}{2}}\exp{(\frac{-\tau\tau_0(\mu_k-\mu_0)^2}{2})}
\end{eqnarray*}
and prior for precision is: \\
\begin{eqnarray*}
p(\tau|\alpha_0,\beta_0)=\frac{\beta_0^{\alpha_0}}{\Gamma(\alpha_0)}\alpha^{\alpha_0-1}\exp(-\beta_0\tau)
\end{eqnarray*}
prior for latent variables is:
\begin{eqnarray*}
p(z|\omega)=\prod_k^K\omega_k^{-z}
\end{eqnarray*}
prior for weights is:
\begin{eqnarray*}
p(\omega|\omega_0)=\prod_{k=1}^K\omega_k^{ \omega_0-1}
\end{eqnarray*}
and finally we set the prior for rotations to constant parameter $ p(R)=1$ .
\\\\
Now we have to compute the conditional posteriors for each of the components.\\
For the latent variable we have:
\begin{eqnarray*}
p(z|x,\mu,\tau,\omega,R)  &\propto &  p(x|z,\mu,\tau,R)p(z|\omega) \\ 
 &\propto & \prod_{ink} N(x_{ik}|P(R_i\mu_k),\tau^{-1}I)^{\beta z_{ink}}\prod_{ink}\omega^{z_{ink}} \\ 
&=& \prod_{ink}[\omega_kN(x_{ik}|P(R_i\mu),\tau^{-1}I]^{\beta z_{ink}}=\prod_{in}p(z_{in}|x_{ik},\mu,\tau,\omega,R) 
\end{eqnarray*}
where
\begin{eqnarray*}
p(z_{in}|x_{in},\mu,\tau,\omega,R)  &\propto & \prod_{k} [\omega_k N(x|P(R\mu_k),\tau^{-1}I)]^\beta  \\
&=& \frac{\prod_{k} [\omega_k N(x|P(R\mu_k),\tau^{-1}I)]^\beta }{\sum_{k} [\omega_k N(x_i|P(R\mu_k),\tau^{-1}I)]^\beta }
\end{eqnarray*} 
The next parameter is weights:
\begin{eqnarray*}
p(\omega | x,z,\mu,\tau,R) &\propto& p(x,z|\omega) p(\omega) \\
&\propto& \prod_{ik}\omega_k^{z_{ik}}\cdot\prod_kw_k^{\alpha_0-1}
\end{eqnarray*}
and for the means we have:
\begin{eqnarray*}
p(\mu|\omega,z,\tau,R) &\propto & p(x,z| \mu,\tau,R)p(\mu | \tau)\\
&\propto & \prod_{ink}N(x_{in} | R_i\mu_k, \tau^{-1}I)^{\beta z_{ink}}\cdot \prod_kN(\mu_k | \mu_0, \tau_0^{-1} \tau^{-1}I)\\
& =& \prod_kp(\mu_k|x,z,\tau,R)
\end{eqnarray*}
where
\begin{equation*}
p(\mu_k|x,z,\tau,R) \propto \prod_{ink}N(x_{in}|R_i\mu_k,\tau^{-1}I)^\beta  \cdot N(\mu_k|\mu_0,\tau_0^{-1}\tau^{-1}I)
\end{equation*} 
%from paper...
likelihood for $\mu$ :
\begin{eqnarray*}
\mu_k &\propto & \exp(-\frac{\beta \tau}{2} \sum_{in}z_{ink}\|X_{in}-PR_i\mu_k\|^2) \\
&=& \exp(const + \sum_{in} -\frac{\beta \tau}{2} z_{ink}\|PR_i\mu_k\|^2 + \sum_{in}{\beta \tau}z_{ink}X_{in}^TPR_i\mu_k )\\
&=& \exp( \mu_k^T[\sum_{in}-\frac{\beta \tau}{2}z_{ink}R_i^{T}P^TPR_i]\mu_k+\mu_k^T\sum_{in}{\beta \tau}z_{ink}R_i^TP^TX_{in})
\end{eqnarray*}
we set: 
\begin{equation*}
A_k=\beta \sum_iN_{ik}(PR_i)^T(PR_i), \qquad N_{ik}=\sum_n z_{ink}, \qquad  b_k=\beta \sum_{in}z_{ink}R_i^TP^TX_{in}
\end{equation*} 
and the posterior is:
\begin{eqnarray*}
\mu_k &\propto & \exp(-\frac{\tau}{2}[\mu_k^TA_k\mu_k-2\mu_k^Tb_k]-\frac{\tau\tau_0}{2}\|\mu_k-\mu_0\|^2) \\
\tilde{A_k} &=& A_k+\tau_0I\\
\tilde{b_k} &=& b_k+\tau_0\mu_0 \\
\mu_k &\propto & \exp(-\frac{\tau}{2}[\mu_k^T\tilde{A_k}\mu_k-2\mu_k^T\tilde{b_k}])\\
\Sigma &=& \tilde{A}_k^{-1}, \qquad \mu=\Sigma \tilde{b_k}
\end{eqnarray*}
and for computing precision we have:
\begin{eqnarray*}
p(\tau|\mu,\omega,Z,R,D) &\propto & p(D,Z|\mu,\omega,R)p(\mu|\tau)p(\tau) \\ 
&\propto & \tau^\frac{Nd}{2}\exp(\frac{-\beta \tau}{2}\sum_{ink}z_{ink}\|x_{in}-PR_i\mu_k\|^2)\\
&\cdot & \tau^\frac{k(d+1)}{2} \exp(-\frac{\tau\tau_0}{2}\sum_k\|\mu_k-\mu_0\|^2)\\
&\cdot & \tau^{\alpha_0-1}\exp(-\beta_0\tau)
\end{eqnarray*}
by taking the logarithm on both sides:
\begin{eqnarray*}
\log{p(\tau|\mu,\omega,z,R,D)} &=& \sum_{i=1}^N\sum_{k=1}^K \beta z_{ik}\log {\left[ {\frac{\tau}{2\pi}}^\frac{d}{2}\|x_i-PR\mu_k\|^2\right] }+(\alpha_0+1)\log\tau-\beta_0\tau \\
&+& \sum_{k=1}^K\log{\tau^{\frac{d+1}{2}}}-\sum_{k=1}^K\frac{\tau\tau_0}{2}\|\mu_k\|^2 \\
&=& \frac{\beta dN}{2}\log{}\tau-\frac{\tau}{2}\sum_{n=1}^N\sum_{k=1}^K \beta z_{ik}\|x_i-PR_i\mu_k\|^2+(\alpha_0-1)\log{\tau}-\beta_0\tau \\
&+& \frac{dK}{2}\log{\tau}-\tau\tau_0\sum_{k=1}^K\|\mu_k\|^2 \\
&=& \left[\frac{\beta dN+(d+1)K}{2}+\alpha_0-1\right]\log\tau-\left[\frac{1}{2}\sum_{n=1}^N\sum_{k=1}^K \beta z_{ik}\|x_i-PR\mu_k\|^2+\beta_0-\tau_0\sum_{k=1}^K\|\mu\|^2 \right]\tau \\
&=& (\tilde{a}-1)\log\tau-\tilde{b}\tau
\end{eqnarray*}
which implies that:
\begin{equation*}
p(\tau|\mu,\omega,z,R,D)=Gamma(\tau| \tilde{a}, \tilde{b})
\end{equation*}
where
\begin{equation*}
2\tilde{a}=2\alpha_0+\beta dN+(d+1)K \\
\end{equation*}
\begin{equation*}
2\tilde{b}=2\beta_0+\sum_{i=1}^N\sum_{k=1}^K \beta z_{ik}\|x_i-PR\mu_k\|^2+\tau\sum_{k=1}^K|\mu_k\|^2
\end{equation*}
and finally the rotations:
\begin{eqnarray*}
p(R|x,z,\mu,\tau,\omega) &\propto & p(x,z|\mu,\tau,\omega,R)p(R)\\
&\propto & \prod_{ink}[\omega_kN(x_{in}|R_i,\mu_k,\tau^{-1}I)]^{\beta z_{ink}}\\
&=& \prod_{i}p(R_i|x,z,\mu,\tau,\omega)
\end{eqnarray*}
where
\begin{eqnarray*}
p(R_i|x,z,\mu,\tau,\omega) &\propto & \exp(-\frac{\tau}{2}\sum_{nk} \beta z_{ink}\|x_{in}-PR_i\mu_k\|^2)\\
& \propto & \exp(\tau tr([\sum_{nk}\beta z_{ink}\mu_kx_{in}^TP]R_i))-\frac{\tau}{2}tr(\sum_{n,k} \beta z_{ink}\mu_k\mu_k^TR_i^TP^TPR_i)\\
A_i&=& \tau\sum_{nk}\beta z_{ink}\mu_kx_{in}^TP=\sum_k(\sum_nx_{in}^T \beta z_{ink})\mu_k\\
C &=& P^TP\\
B_i &=& -\frac{\tau}{2}\sum_{nk}\beta z_{ink}\mu_k\mu_k^T=-\frac{\tau}{2}\sum_{k}\beta N_{ik}\mu_k\mu_k^T\\
\end{eqnarray*}
where
\begin{equation*}
 N_{ik}=\sum_nz_{ink}
\end{equation*}
you can find the numerical computational process for the rotations in appendix. \\
General form of the likelihood for GMM:
\begin{equation*}
p(D|\mu,\tau,\omega,R) = \prod_{n=1}^N\sum_{k=1}^k\omega_kN(x_n|PR\mu_k,\tau)
\end{equation*}
The Augmented likelihood is:
\begin{eqnarray*}
p(D|\mu,\tau,z,R) &=& \prod_n\prod_k[N(X_n|\mu_k,\tau)]^{\beta z_{nk}}\\
p(z|\omega) &=& \prod_nM(z_n|1,\omega)=\prod_n\prod_k\omega^{z_{nk}}\delta(1-\sum_kz_{nk})
\end{eqnarray*}
 [ if we set, $\omega_k=\frac{1}{K} $, \quad we will have:
\begin{eqnarray*}
\log{p(z|\omega)}&=&\sum_n\sum_kz_{nk}\log{w_k} \\
&=&\log{\omega}\sum_n\sum_kz_{nk} \\
&=&N\log{\omega}=-N\log{K}]
\end{eqnarray*}  
\begin{eqnarray*}
M(n|N,P) &=& (\frac{{N!}}{n_{1!},...,n_{k!}}\prod_kP_k^{n_k}\delta(N-\sum_kn_k)) \\
p(D|\mu,\tau,\omega,R) &=& \sum_zp(D|\mu,\tau,z,R)p(z|w)\\
p(D,z|\mu,\tau,\omega,R) &=& p(D|\mu,\tau,z,R)p(z|\omega)\\
\end{eqnarray*}
The likelihood and the priors in total:
\begin{equation*}
\left[p(D|\mu,\tau,z,R)p(z|\omega)\right]\left[p(\omega|\omega_0)\prod_kp(\mu_k|\mu_0,\tau_0,\tau)p(\tau|\alpha,\beta)p(R)\right]
\end{equation*}

\section{Appendix}
Computation of the Rotations :
\begin{eqnarray*}
tr(BR^TCR)+tr(AR)
\end{eqnarray*}
where
\begin{eqnarray*}
B^T &=& B, \qquad -B>0, \qquad X^T(-B)X\geq0 \\
-B &=& U\Lambda U^T \\
\Lambda &=& \begin{bmatrix}\lambda_1 & 0 &0 \\0 & \lambda_2 &0 \\ 0 &0 & \lambda_3\end{bmatrix}
\end{eqnarray*}
and
\begin{eqnarray*}
 \lambda_i\geq0, \qquad U^TU=UU^T=I_{3\times3} 
\end{eqnarray*}
so we have:
\begin{eqnarray*}
-tr(BR^TCR)+tr(AR)&=&tr(U\Lambda U^TR^TCR)+tr(ARUU^T) \\
&=&tr(\Lambda (RU)^TC(RU))+tr((U^TA)(RU)) \\
&=&tr(\Lambda \tilde{R}^TC\tilde{R})+tr(\tilde{A}\tilde{R}) \\
\end{eqnarray*}
where
\begin{eqnarray*}
\tilde{R}=RU, \qquad \tilde{A}=U^TA 
\end{eqnarray*}
and  
\begin{eqnarray*}
\tilde{R}^T\tilde{R}=I 
\end{eqnarray*}
and 
\begin{eqnarray*}
C=P^TP=\begin{bmatrix}1 & 0 &0 \\ 0 & 1 & 0 \\ 0&0&0 \end{bmatrix}=I-e_3e_3^T 
\end{eqnarray*}
\begin{eqnarray*}
tr(\Lambda \tilde{R}^T(I-e_3e_3^T)\tilde{R}) &=& tr(\Lambda)-tr(\Lambda\tilde{R}^Te_3e_3^T\tilde{R}) \\
&=&tr(\Lambda)-e_3^T\tilde{R}\lambda\tilde{R}^Te_3 \\
&=&tr(\Lambda)-\tilde{r}_3^T\Lambda\tilde{r}_3 
\end{eqnarray*}
where
\begin{eqnarray*}
 \tilde{r}_i &=& \tilde{R}^Te_i, \qquad \tilde{r}_i\in R^3 \\
tr(AR) &=& tr(\tilde{A}\tilde{R}) 
\end{eqnarray*}
where
\begin{eqnarray*}
 \tilde{A} &=& U^TA, \qquad A=D_{3\times2}P_{2\times3} \\
D &=& \tau\sum_{nk}z_{ink}\mu_kX_{in}^T \\
\tilde{A} &=& U^TDP, \qquad \tilde{D}=U^TD\\
\tilde{A} &=& \tilde{D}P 
\end{eqnarray*}
where
\begin{eqnarray*}
 P &\in & R^{2\times3}, \qquad D\in R^{3\times2} \\
tr(AR) &=& tr(\tilde{D}P\tilde{R}) \\
P\tilde{R} &=& \begin{bmatrix}1 & 0&0 \\ 0 & 1&0 \\ \end{bmatrix}\left(\begin{array}{c}\tilde{r_1}^T\\ \tilde{r_2}^T \\ \tilde{r_3}^T\end{array}\right)=\left(\begin{array}{c}\tilde{r_1}^T\\ \tilde{r_2}^T \\ \end{array}\right) \in R^{2\times3} \\
\tilde{D}^T &=& \left(\begin{array}{c}\tilde{d_1}^T\\ \tilde{d_2}^T \\ \end{array}\right) \\
tr(AR) &=& tr(\tilde{D}\left(\begin{array}{c}\tilde{r_1}^T\\ \tilde{r_2}^T \\ \end{array}\right))=\tilde{d_1}^T\tilde{r_1}^T+\tilde{d_r}^T\tilde{r_2}^T 
\end{eqnarray*}
so finally we have:
\begin{eqnarray*}
-tr(BR^TCR) &+& tr(AR) \\
&=& tr\Lambda-\tilde{r_3}^T\Lambda\tilde{r_3}+\tilde{d_1}^T\tilde{r_1}^T+\tilde{d_2}^T\tilde{r_2}^T 
\end{eqnarray*}
subject to,
\begin{eqnarray*}
 \tilde{r_i}\bot\tilde{r_j}
\end{eqnarray*}


\end{document}